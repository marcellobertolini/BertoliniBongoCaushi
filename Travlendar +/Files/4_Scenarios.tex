\mysection{Scenarios}

\mysubsection{Scenario 1}
Giuseppe is an employee of a society. One night, before going to sleep, he receives a call from his chief that inform him that he has to attend three meetings in three different zone of Milan. The next day the chief needs to know if Giuseppe can take part at all these meetings or if he has to ask to someone else. Giuseppe has never been in those zones of Milan and, at the moment, he doesn’t have his car because it’s broken. Thus, he has to find another solution. 
In that moment an idea came in Giuseppe’s mind! He opened the Travlendar+ app on his smartphone and inserted the three events in the calendar using a TAG called \emph{no car}, which was specifically created for this situation. This tag includes all the means of transport supported by the app except the car. During the creation of the last event, Travlendar+ has showed to Giuseppe a warning telling him that the event was unreachable in time from the previous one he set. Knowing this information, Giuseppe called his chief for telling him that he has to ask to someone else just for the third event and that he can attend the first two without any problems because he will take the means suggested by Travlendar+.

\mysubsection{Scenario 2}
Michela is a very busy person and she cares about the environment. She has been very happy to download Travlendar+ on her smartphone because it can suggest the means of transport that have the less impact on the environment. Thus, every day Michela inserts her events in Travlendar+ and she associates to each one the \emph{CO$_2$ free} preference. In this way, the system gives her the information about all the means of transport available for reaching the destination, order by the one who produce the less CO$_2$ quantity to the one who produce the most. This is very useful for Michela, because now she can both follow her wish of having a good impact on the environment and avoid arriving late to her appointments, finding always a good compromise without any difficulties.

\mysubsection{Scenario 3}
Luca lives in Milan and the 20$^{th}$ of October he has a work meeting in Turin. He decides to insert the meeting in his calendar. As soon as he sets all the data defining the event and concludes the creation, a message appears saying that the event is very far and advices Luca to plan a trip. Luca is very happy of having this opportunity and proceeds on. Travlendar+ provides him an interface in which he can chose of buying a ticket for the train or for the airplane or simply adding a travel event with that mean. Luca selects the train, then Travlendar+ asks Luca to fill up a form containing the date and the time in which he would like to leave, in order to provide him all the solutions with the related starting prices. After a quick look, thanks to the ad-hoc filtration of the trains, Luca is able to choose the most suitable ticket. As soon as he decides to purchase it, two things happen: 
\begin{itemize}
	\setlength{\leftskip}{0.5cm}
	\item a travel event with all the information is automatically created;
	\item Luca is addressed on the travel service website to complete the purchase.
\end{itemize}
Luca will be able to check the solutions for arriving to the station on time in every moment tapping on the travel event auto-generated.
When Luca returns on the application, he inserts the period of staying and his accommodation in Turin. After this Travlendar+ advices him to buy the ticket that represents the best compromise between price and utility. Anyway, it also shows him the other possible tickets he can buy.

\mysubsection{Scenario 4}
Sean lives in Milan and he has to attend a work-lunch in Rome tomorrow. He has already the ticket for reaching Rome, because his company has already bought it. Thus, he wants to add the travel event in Travlendar+. Nothing of more simple! Sean has only to tap for about one second the screen on the calendar interface and two choices will appear on the screen: \emph{create an event} and \emph{plan a trip}. Tapping on plan a trip, Travlendar+ will show to Sean an interface in which he can chose the mean of transport and, after this, he can simply insert all the info of his ticket and create the travel event tapping on \emph{Add}.

\mysubsection{Scenario 5}
Gianluca is a financial expert, he has always many meetings during the day and sometimes he forgets to take a break. He needs something which reminds him to reserve some time for having lunch. Travlendar+ is the perfect solution to his needs. Gianluca set in the preferences the span of time in which he wants to have lunch, from 12 to 14, and the preferred duration, 60 minutes. 
Today Gianluca has to add to his calendar one event from 12:00 to 13:00 and another one from 13:30 to 15:00. Travlendar+, given the Gianluca’s setting for the lunch, has already automatically created an event lunch every day between from 12:00 and 13.00. When Gianluca adds the first event, the lunch break slides automatically and covers the period between 13.00 and 14.00 without giving any problems. When the second event was added, some problems came up! In fact, Gianluca will not be able to have a lunch of 60 minute. Despite this, Travlendar+ must reserve at least 30 minutes to lunch per day to each user, so it asks Gianluca if he wants to reduce his lunch time to 30 minutes or postpone the event. Gianluca cannot postpone the event, so he decides of reducing his lunch duration to 30 minutes for tomorrow, leaving the other lunch events unchanged.

\mysubsection{Scenario 6}
Gabriele has inserted in his Travlendar+ application his girlfriend tonight’s party event but he has completely forgotten about it. Fortunately, Travlendar+ sends him warnings to remind him to go to the event whenever the time for reaching the party’s location, without arriving late, with the slowest mean of transport in the list is going to expire. But during almost all these warnings Gabriele was sleeping and he did not read them. He just gets the last one, which advices him to use the car. Unfortunately, Gabriele doesn’t have a car in Milan but he has the drive licence and so he can take a car sharing. Travlendar+ is perfect for Gabriele’s situation! In fact, when he taps on the car choice, the app opens an interface that asks Gabriele to choose between his own car and a car sharing. Choosing the second one, Gabriele is able to see the route for reaching the nearest car location and he can also be addressed to the car sharing service app to book it and reaching his girlfriend party’s location on time. Travlendar+ will also show him the fastest route to reach his destination.