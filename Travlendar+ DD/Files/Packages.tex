\documentclass [11pt,a4paper]{article}
\usepackage[utf8]{inputenc}
\usepackage[T1]{fontenc}
\usepackage[english]{babel}
\usepackage{textcomp}
\usepackage{lmodern}

% Text Indentation
\usepackage{indentfirst}
\setlength{\parindent}{1cm}

% Section, Subsection & Subsubsection Indentation
\newcommand{\mysection}[1]{
	\setlength{\leftskip}{0cm}
	\section{#1}}
\newcommand{\mysubsection}[1]{
	\setlength{\leftskip}{0.5cm}
	\subsection{#1}}
\newcommand{\mysubsubsection}[1]{
	\setlength{\leftskip}{1cm}
	\subsubsection{#1}}

% Defined header and footer style
\usepackage{fancyhdr}
\pagestyle{fancy}
\renewcommand{\sectionmark}[1]{%
	\markright{\thesection\ #1}}
\fancyhf{}
\fancyhead[LE,RO]{\bfseries\thepage}
\fancyhead[LO]{\bfseries\rightmark}
\fancyhead[RE]{\bfseries\leftmark}
%\lhead{\color{Gray}{\small{Travlendar+ : Manage your Day !}}}
\lfoot{\textcolor{Gray}{\small{Copyright © 2017 Kostandin Caushi, Marcello Bertolini, Raffaele Bongo – All rights reserved}}}
%\rfoot{\textcolor{Gray}{\thepage}}
\renewcommand{\headrulewidth}{1pt}
\fancyhfoffset[lf]{0mm}

% Page margins, header and footer positions
\usepackage{geometry}
\geometry{
	a4paper,
	total={210mm,297mm},
	left=25mm,
	right=25mm,
	top=35mm,
	bottom=25mm,
	headsep=10mm
	}

\interfootnotelinepenalty=10000

% Tables
\usepackage{tabu}
\usepackage{tabularx}
\usepackage{multirow}
\usepackage{ltablex}
\usepackage{longtable}
\usepackage{float} % To allow the use of H modifier in long tables

% Graphics
\usepackage{graphicx}
\usepackage[dvipsnames, table]{xcolor}

\usepackage{ifthen}
\usepackage{xspace}
\usepackage{enumitem}
\usepackage{amssymb}
\usepackage[pdftex, colorlinks]{hyperref}
\hypersetup{%
	colorlinks = true,
	linkcolor  = black,
	pdfauthor  = {Kostandin Caushi, Marcello Bertolini, Raffaele Bongo},
	pdftitle   = {Travlendar+}
}
\newcommand{\comment}[1]{{\color{Blue}$\blacktriangleright$ Comment: #1 $\blacktriangleleft$}}

% Java Package
\usepackage{listings}
\usepackage{color}

\definecolor{mygreen}{rgb}{0,0.6,0}
\definecolor{mygray}{rgb}{0.5,0.5,0.5}
\definecolor{mymauve}{rgb}{0.58,0,0.82}

\lstset{ %
	backgroundcolor=\color{white},   % choose the background color; you must add \usepackage{color} or \usepackage{xcolor}; should come as last argument
	basicstyle=\footnotesize,        % the size of the fonts that are used for the code
	breakatwhitespace=false,         % sets if automatic breaks should only happen at whitespace
	breaklines=true,                 % sets automatic line breaking
	captionpos=b,                    % sets the caption-position to bottom
	commentstyle=\color{mygreen},    % comment style
	deletekeywords={...},            % if you want to delete keywords from the given language
	escapeinside={\%*}{*)},          % if you want to add LaTeX within your code
	extendedchars=true,              % lets you use non-ASCII characters; for 8-bits encodings only, does not work with UTF-8
	frame=single,	                   % adds a frame around the code
	keepspaces=true,                 % keeps spaces in text, useful for keeping indentation of code (possibly needs columns=flexible)
	keywordstyle=\color{blue},       % keyword style
	language=Octave,                 % the language of the code
	morekeywords={*,...},            % if you want to add more keywords to the set
	numbers=left,                    % where to put the line-numbers; possible values are (none, left, right)
	numbersep=5pt,                   % how far the line-numbers are from the code
	numberstyle=\tiny\color{mygray}, % the style that is used for the line-numbers
	rulecolor=\color{black},         % if not set, the frame-color may be changed on line-breaks within not-black text (e.g. comments (green here))
	showspaces=false,                % show spaces everywhere adding particular underscores; it overrides 'showstringspaces'
	showstringspaces=false,          % underline spaces within strings only
	showtabs=false,                  % show tabs within strings adding particular underscores
	stepnumber=2,                    % the step between two line-numbers. If it's 1, each line will be numbered
	stringstyle=\color{mymauve},     % string literal style
	tabsize=2,	                   % sets default tabsize to 2 spaces
	title=\lstname                   % show the filename of files included with \lstinputlisting; also try caption instead of title
}