\mysection{Implementation, Integration and Test Plan}
In this section we will show the plan we have projected for implementing Travlendar+. 
Firstly we will describe the strategy for the processes chosen for taking the project and the structure of our team, how it is divided and the tasks that each part has to accomplish.
After we will state how each team’s part has to interact with the others and the span of time of these interactions. 
Finally we will supply a list of possible risks, the probability of their presence, their impact and a possible strategy to deal with them.

\mysubsection{Strategy adopted}
To give a quality assurance of the project and to be sure that the final product will be as our stakeholders expect, we will follow an Agile planning process. It consists in a first initiating part in which it’s given an overall plan of the system process. We have started to build the plan with the information given in the RASD and we will terminate it in this document: in fact in the previous sections we have described the software architecture and the design patterns that will be used and in this section we will give a schedule for all the various team tasks.
\begin{figure}[H]
	\centering
	\includegraphics[scale=0.35]{Images/Implementation/Agile_Planning}
	\caption{Agile Planning Strategy}
\end{figure}
After this part, there will be a cycle of phases called respectively: Executing, monitoring\&Controlling, Closing and Planning. 
Every cycle round has as input a specific process to accomplish, that is divided between the various team’s parts. After the execution of all the tasks in a precise given timeframe, the work is checked by all the parts, customers included. When an agreement is reached, it will follow another planning phase that will contain also the corrections that will result after the agreement.
This cycle is repeated until the end of the project, when all the functionalities stated in the RASD will be covered.

\newpage
\mysubsection{Team structure}
Our team will be composed by five main parts :
\begin{itemize}
	\item \emph{Developers :} people devoted to implement the software.
	
	\item \emph{Testers :} people devoted to test the software functionalities.
	
	\item \emph{Front-End Team :} developers appointed to implement the system’s front end.
	
	
	\item \emph{Back-End Team :} developers appointed to implement the system application logic.
	
	\item \emph{Supervisors :} developers that belongs to the Back-End Team or to the Front-End Team and that have the main task of having any time the picture of all the their team work.
\end{itemize}

\mysubsection{Implementation and testing plan}
The work will be divided in three milestones. In these span of time all the teams have to complete the tasks that have been established in the previous planning phase. Every two days the different team supervisors and a Tester’s team delegate will organize a meeting to discuss about their team progress and they will give directives to proceed on basically at the same speed. If a team’s part terminate his tasks before the milestone’s day, it can anticipate the work programmed for the subsequent milestone, but remaining ready for eventually applying changes or corrections on the software already implemented.
In the meanwhile, the Tester team will work on the \emph{white-box} units test for the functionalities that the developer’s teams are implementing.
When the milestone day comes, the Tester team will have from a minimum of one day to a maximum of three days for testing the functionalities and give the results to the development team that will have to fix the eventual bugs.
When the unit test is passed, the Tester team has to perform an integration test of the components and give the result to the development teams which have to fix eventual integration problems and, except for the first milestone, release a beta version to a restrict group of selected people.

In this way, it will be performed also an incremental \emph{User acceptance test} on the work that has been performed. Furthermore, it will take place a meeting with the customers in which their needs and doubts will be discussed and taken into account in the next planning phase.

The precise duration of the timeframe will be established dynamically taking into account the minimum and maximum span on time mentioned above considering that the project must be completed in no more of 16 weeks.

\begin{figure}[H]
	\centering
	\includegraphics[scale=0.22]{Images/Implementation/Implementation_Flow}
	\caption{Implementation Flow}
\end{figure}

\mysubsection{Possible risks}
\begin{tabular}[H]{p{7cm}|c|c}
	Risk & Probability & Effects\\
	\hline
	\rule{0pt}{4ex} Budget Problems & Low & Catastrophic\\
	\hline
	\rule{0pt}{4ex} Time for Implementing a Function Exceed & Medium & Serious\\
	\hline
	\rule{0pt}{4ex} Team’s Members not Available in a Critic Moments & Medium & Catastrophic\\
	\hline
	\rule{0pt}{4ex} Impossibility to Recruit Staff with Required Skills & Medium & Serious
\end{tabular}

\newpage
\mysubsection{Possible solutions}
\begin{itemize}
	\item \emph{Budget problems :}specify before the beginning all the functionalities, realize prevision of the costs and agree with the customers a margin of budget that could increase due to unpredicted situation during the development.
	
	\item \emph{Time for implementing a function exceed :}agree with the stakeholder a span of time of delay that they can accept for unpredicted problems.
	
	\item \emph{Team’s member not available in critic moment :} being sure that at least one of the supervisors could substitute a team member in case of necessity
	
	\item \emph{Impossibility to recruit staff with required skills :} advices the customers about possible delays and to be ready with a table of COTS to buy in case of needs with the relative costs.
	
\end{itemize}