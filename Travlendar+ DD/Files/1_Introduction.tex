\vspace*{-5mm}
\mysection{Introduction}

\mysubsection{Purpose}
This document has the aim of entering into the details of Travlendar+ system. 
We will show the software architecture that we have designed for our system with different levels of abstraction, analysing deeply the main components. 
Additionally, we will exhibit a runtime view of the system, showing as the various components will interact with each other, specifying architectural and design pattern used.
Finally, we will present the critical algorithm implemented, a requirement - software components correspondence and a high level plan about the implementation and integration of the various components.

\mysubsection{Scope}
Travlendar+ is a calendar-based application that has the aim of managing the many meetings, events and appointments that a user has to deal with every day.\par
The system will let the user create events in his personal calendar, checking if he is able to reach them on time and supporting his choices about the way of reaching the location. \par
Travlendar+ will also give the user the possibility to buy tickets of a town’s public and private means of transport and it will also allow him to manage his travels to reach other cities, creating specific travel events in the calendar section.
The system will offer other additional features :
\begin{itemize}
	\setlength{\leftskip}{0.5cm}
	\item The possibility to register the season ticket for the public transport. Travlendar+ will notify the user when the expiry date is near.
	\item The possibility to set the starting time, ending time and the preferred duration of every day lunch. The system will guarantee to reserve at least 30 minutes for this purpose.
	\item In case of outdoor trips, the user will be able to insert the period he will spend out of town and the system will suggest him the most convenient transport tickets available, keeping in mind the information given.
	\item The possibility of setting the anticipation time for reaching the various events. The system will warn the user when he needs to leave in order to arrive on time.
\end{itemize}

\newpage
\mysubsection{Definitions}
\begin{itemize}
	\item \emph{Overlap :} if there’s an event previously planned in that specific moment.
	\item \emph{Reachability :} if an event is reachable from the previous one.
	\item \emph{Tier :} a physical structuring mechanism for the system infrastructure.
	\item \emph{Server :} a computer program or a device that provides functionalities for other programs or devices, called clients.
	\item \emph{Interface :} is a shared boundary across which two or more separate components of a computer system exchange information.
	\item \emph{Design Pattern :} it's a general reusable solution to a common problem that occurs in a given context in software design.
	\item \emph{Milestone :} word used to indicate an established day. Within this day the various teams have to reach a determined result.
	\item \emph{Front-End :} it's a part of the system with which the user interact directly.
	\item \emph{Back-end :} it's a part of the system that contains all the system logic and interacts with the Data Bases and the external software services, but it's hidden to the user.
\end{itemize}

\mysubsection{Acronyms}
\begin{itemize}
	\setlength{\leftskip}{0.5cm}
	\item \emph{RASD :} Requirement Analysis and Specification Document
	\item \emph{API :} Application Programming Interface
	\item \emph{SOA :} Service Oriented Architecture
	\item \emph{MVC :} Model View Controller
	\item \emph{HTTP :} HyperText Transfer Protocol
	\item \emph{HTML :} HyperText Markup Language
	\item \emph{DB :} Database
	\item \emph{REST :} REpresentational State Transfer
\end{itemize}

\mysubsection{Abbreviations}
\begin{itemize}
	\setlength{\leftskip}{0.5cm}
	\item \lbrack Gn] : Goal n
	\item \lbrack Rn] : Requirement n
\end{itemize}

\mysubsection{Reference Documents}
\begin{itemize}
	\setlength{\leftskip}{0.5cm}
	\item Mandatory Project Assignments.pdf
	\item Travlendar+ RASD
	\item Project Management part 1.pdf
	\item Project Management part 2.pdf
	\item Verification and Validation.pdf
	\item The Component Diagram - IBM developerWorks
\end{itemize}

\mysubsection{Document Structure}
This paper is divided in 7 chapter:
\begin{enumerate}
	\setlength{\leftskip}{0.5cm}
	\item The first chapter is composed by an introduction of the system, its application domain, its goals and a glossary containing the most common expression used in order to give to the reader a basic knowledge of the system and to make him understand better the subsequent parts.
	
	\item In the second chapter it is defined the system architecture. Starting from a High level elements description, it goes deeper step by step inside the various system components and their interactions. The architecture styles and the design patterns that we have used in designing our system are also shown in the last subsections.
	
	\item The third paragraph is focused on the algorithm we have designed to perform all the system check operations. The problem we faced and its solution are precisely described in this paragraph and a Java code example is also provided.
	
	\item In the fourth paragraph some mockups, explaining the user interactions with the system, are shown and described.
	
	\item In the fifth chapter is shown how the requirements we have defined in the RASD map to the design elements that we have defined in this document.
	
	\item In the sixth chapter a strategy to address the Travlendar+ implementation is provided. Furthermore, we have defined a precise team structure describing the interactions between the various team parts. Finally, we have defined a table of the possible risks that may occur during the development process and a list of the possible solutions.
	
	\item In the seventh and final chapter there is the list of programs used, the document versions and the authors effort spent.
\end{enumerate}