\vspace*{-5mm}
\mysection{Introduction}

\mysubsection{Purpose}
This document has the aim of entering into the detail of Travlendar+ system. 
We will show the software architecture that we have designed for our system with different levels of abstraction, analyzing deeply the main components. 
Additionally we will exhibit a runtime view of the system, showing as the various components will interact between themself specifying architectural and design pattern used.
Finally we will present the critical algorithm implemented, a requirement - software componets correspondency and an high level plan about implementig and integrating the various components.

\mysubsection{Scope}
Travlendar+ is a calendar-based application that has the aim of managing the many meetings, events and appointments that a user has to deal with every day.\par
The system will let the user create events in his personal calendar, checking if he is able to reach them on time and supporting his choices about the way of reaching the location. \par
Travlendar+ will also give the user the possibility to buy tickets of a town’s public and private means of transport and it will also allow him to manage his travels to reach other cities, creating specific travel events in the calendar section.
The system will offer other additional features :
\begin{itemize}
	\setlength{\leftskip}{0.5cm}
	\item The possibility to register the season ticket for the public transport. Travlendar+ will notify the user when the expiry date is near.
	\item The possibility to set the starting time, ending time and the preferred duration of every day lunch. The system will guarantee to reserve at least 30 minutes for this purpose.
	\item In case of outdoor trips, the user will be able to insert the period he will spend out of town and the system will suggest him the most convenient transport tickets available, keeping in mind the information given.
	\item The possibility of setting the anticipation time for reaching the various events. The system will warn the user when he needs to leave in order to arrive on time.
\end{itemize}

\newpage
\mysubsection{Definitions}
\begin{itemize}
	\item \emph{Overlap :} if there’s already a planned event in that specific moment.
	\item \emph{Reachability :} if an event is reachable from the previous one.
	\item \emph{Tier :} physical structuring mechanism for the system infrastructure.
	\item \emph{Server :} a computer program or a device that provides functionality for other programs or devices, called clients.
	\item \emph{Interface :} is a shared boundary across which two or more separate components of a computer system exchange information.
	\item \emph{Design Pattern :} is a general reusable solution to a commonly occurring problem within a given context in software design.
	\item \emph{Milestone :} checkpoint of the implementation process. The various team have to reach determined result for the milestone's day.
	\item \emph{Sub-Milestone :} checkpoint that together produce the milestone.
	\item \emph{Front-End :} part of the system with which the user directly interact.
	\item \emph{Back-end :} part of the system that contains all the system's logic and intercat with the Data Bases and the external services' software but that is hidden to the user.
\end{itemize}

\mysubsection{Acronyms}
\begin{itemize}
	\setlength{\leftskip}{0.5cm}
	\item \emph{RASD :} Requirement Analysis and Specification Document
	\item \emph{API :} Application Programming Interface
	\item \emph{SOA :} Service Oriented Architecture
	\item \emph{MVC :} Model View Controller
	\item \emph{HTTP :} HyperText Transfer Protocol
	\item \emph{HTML :} HyperText Markup Language
	\item \emph{DB :} Database
	\item \emph{REST :} REpresentational State Transfer
\end{itemize}

\mysubsection{Abbreviations}
\begin{itemize}
	\setlength{\leftskip}{0.5cm}
	\item \lbrack Gn] : Goal n
	\item \lbrack Rn] : Requirement n
\end{itemize}

\mysubsection{Reference Documents}
\begin{itemize}
	\setlength{\leftskip}{0.5cm}
	\item Mandatory Project Assignments.pdf
	\item Travlendar+ RASD
	\item Project Management part 1.pdf
	\item Project Management part 2.pdf
	\item Verification and Validation.pdf
	\item The Component Diagram - IBM developerWorks
\end{itemize}

\mysubsection{Document Structure}
This paper is divided in 6 chapter:
\begin{enumerate}
	\setlength{\leftskip}{0.5cm}
	\item The first chapter is constituted by an introduction of the system, his application domain, his goals and a glossary containing the most common expression used in order to give to the reader a basic knowledge of the system and to make him understand better the subsequent parts.
	
	\item In second chapter is defined the system's architecture. This is done starting from an High level elements description and going step by step deeper inside the various components and their interactions. The architeture styles and the design patters that we exploited for our design are also showed in the last subsections.
	
	\item In the third paragraph there is a focus on the algorithm that we have designed for performing all the our system's check operations. The problem that we took on and its solution are precisely described and a code example in Java is also provided.
	
	\item In the fourth paragraph some mockups that describes user's interactions with the system are showed with a precice description of the interactions.
	
	\item In the fifth chapter is shown how the requirements we have defined in the RASD map to the design elements that we have defined in this document.
	
	\item In the sixth chapter we have provided a startegy for dealing with Travlendar+ implementation. Furthemore, we have stated a precise team structure with the interactions between the various team's parts. Finally, we have stated a table of possible risks that may occur during the development process and a list of possible solutions.
\end{enumerate}